\section{Fractional Iterates of the Exponential Function} \label{sec:frac}

\subsection{Abel's Functional Equation}

Assuming $\psi(x) = x$ for $0 \leq x < 1$, we can find a solution for $\psi$ which is valid on all of $\setR$. 
Since we have $\log(x) < x$ for any $x \geq 1$, multiple applications of \eqref{eq:abellog} will, at some point, lead to an argument for $\psi$ that falls into the interval where we assume $\psi$ to be linear. 
In case $x<0$, we can use $\psi(x) = \psi(\exp(x)) -1$ to reach the desired interval, since $0 < \exp(x) < 1$ for $x < 0$. 
In combination, this leads to a piece-wise defined solution for $\psi$,
\begin{subequations}
\begin{gather}
\label{eq:psi}
\psi(x) = log^{(k)}(x) + k \\
\label{eq:psicond}
\text{with }k \in \setN \cup \{-1,0\}: 0 \leq \log^{k} (x) < 1 \,.
\end{gather}
\end{subequations}
The function is displayed in Fig.~\ref{fig:psi}.

\begin{figure}
\centering
\includegraphics[width=0.8\columnwidth]{figures/psi.tikz}
\caption{A continuously differentiable solution $\psi(x)$ for Abel's equation \eqref{eq:abel} with $f(x) = \exp(x).$}
\label{fig:psi}
\end{figure}